\section{Introduction} \label{Introduction}

Is it possible to reduce greenhouse gas (GHG) emissions and create jobs at the same time? The double challenge of global warming and unemployment has motivated a long quest for such a panacea. And it remains an acute topic as global warming is likely to exceed 2°C by the end of the century and unemployment persists at high post-crisis levels in many countries in Europe and elsewhere.

The urgent need to take action and mitigate global warming is unanimously recognised by the international scientific community. To make substantial and sustained reductions in greenhouse gas emissions, we must transform the way we produce and use energy, through investment in renewable sources of energy, energy efficiency and new infrastructure. The USD 1.8 trillion invested in energy each year \citep{IEAWIR2016} needs to be redirected towards renewable sources. The importance of such a shift was restated in the Paris Agreement: Article 2 recalls the importance of "making finance flows consistent with a pathway towards low greenhouse gas emissions and climate-resilient development". 

Such sums of money have attracted attention with respect to their impacts on the sensitive issue of employment. In various policy communications and media, the potential for so-called “green jobs” is used as a central argument in favour of renewable energy or weatherproofing programs – sometimes even placed above their benefits for the climate. These investments would create jobs because they are more local and labour-intensive than burning fossil fuels in capital-intensive plants. But similar arguments on jobs were also used by Donald Trump to leave the Paris Agreement.\footnote{He insisted on the importance of coal-related jobs in the US. The word "job" was used 18 times in his speech.}

Considering employment impacts as a simple secondary benefit of global warming mitigation would be a mistake. Today, this second dividend is a powerful lever to initiate public action. Although the Paris Agreement has made countries' emissions pledges more ambitious, international negotiations still stumble on a version of the prisoner's dilemma: individually, each nation might try to free-ride as much as possible and let the other countries bear the bulk of the climate burden. This partly explains why the sum of INDC from the Paris Agreement is far from meeting its overall target of staying "well below 2°C"\footnote{According to the \citet{OECD/IEA2016}'s World Energy Outlook, the Paris Agreement pledges are "not nearly enough to limit warming to less than 2°C"}, while at the same time the IPCC chair, Dr Pachauri, considers that “the solutions are many and allow for continued economic and human development. All we need is the will to change”\footnote{Declaration at the release of the Fifth Assessment Synthesis Report of the IPCC, November 2, 2014}.
As long as ecology and economy are conceived in terms of trade-offs, such difficulties are doomed to continue. The issue of employment is a local matter which can accelerate, as well as hinder, the energy transition transformation. 

But is it possible to generate net positive job creation by simply shifting demand towards low-carbon sectors? 
The literature on this dual topic of environment and economy is known as the "double dividend" debate. It studies under what conditions protection of the environment (the first dividend) can be obtained in conjunction with an economic benefit (the second dividend).
Results from this literature are still mixed, and it is difficult to find robust conclusions. Two main reasons explain this difficulty. The first one is the multiplicity of situations studied: analyses refer to different countries, with various scenarios and data hypotheses on technology costs or production structures. The second reason relates to the variety of tools employed: different economic models are used. In particular, two main families of economic models coexist in the energy-employment literature: Input-Output (IO) models and Computable General Equilibrium (CGE) models. Moreover, it is difficult to disentangle which results stem from the model and which do not. Yet, these two types of model continue to be used in academic publications and reports to policy makers - without comparison between the two.

Our aim is thus to understand the economic mechanisms of job creation due to investment shifts. Why would divesting ourselves of fossil fuels and favouring sectors that will encourage a low-carbon future, such as renewable energy or building insulation, create jobs? Do the arguments about local jobs, domestic sources of energy and labour-intensive technologies hold in a general equilibrium?
Our analysis should also enable us to determine robust results across both types of model, IO and CGE, and to highlight their differences. Do these two types of model yield approximately similar, very different, or even completely opposite results? Answering these questions is necessary in order to test the validity of that part of the literature on green jobs that uses IO models.

Our work is related to the this literature, for which there are many quantitative estimates, using both CGE models \citep{Lehr2008, Lehr2012, Bohringer2013, Blazejczak2014, Creutzig2014, Duscha2014, Duscha2016, Chen2016} and IO models \citep{Hillebrand2006, Scott2008, deArce2012, Markaki2013, Hartwig2016, Yushchenko2016, Li2016, Garrett2017}.
We also use the methodology that \citet{Garrett2017} recently formalised and coined as the "synthetic industry approach" to estimate investment requirements for low-carbon sectors. This new approach helps circumvent the issue of data availability with respect to renewable energies, which was highlighted by \citet{Cameron2015}.
Finally, we also connect to the small strand of literature comparing IO and CGE models, although this work was published outside the energy field. These papers study the employment impacts of an increase in investment, and pointed out that IO models yield higher figures than their CGE counterparts \citep{Partridge1998, OHara2013, Dwyer2005}.

However, our paper has two original aspects. 
First, we go beyond a simple quantification, and disentangle the mechanisms of job creation at play when investment is shifted. 
Our analysis starts with the simpler IO model. We show that, within this framework, the potential for job creation depends on three parameters: the share of labour in value added, the level of sectoral wages and the rates of imports. Shifting investment generates jobs if the shift targets sectors with a higher share of labour in value added, with lower wages or with a lower rate of imports. After a general discussion on these three levers, we test them one by one, using several stylised CGE models, each designed to test a specific assumption.
Breaking down job creation into three levers, and studying each of these levers separately has not yet been done to our knowledge, either for a CGE or an IO model, let alone for both models together.
Second, we provide quantitative employment estimates of investing in solar PV and weatherproofing, using both an IO model and a fully-fledged CGE. Our two models are based on exactly the same data, at a disaggregated level (58 sectors), with the most recent values available for national accounts (2013). To our knowledge, such a comparison exercise has not yet been undertaken in the energy field. Moreover, our results might differ from the few papers doing the same kind of quantitative comparison in other fields, as the local and labour-intensive characteristics of renewable energies are often quoted as being key to their positive employment potential.

Our analysis indicates that it is possible to generate employment in a CGE by shifting demand towards sectors with a high labour share or low wages. On the contrary, we see no positive impact of targeting sectors with low import rates in a CGE, conversely to many political messages.
IO models provide a good approximation of the impacts of sectoral wage differences on employment. They also provide the right direction for the labour-intensity effect. But they diverge significantly as to the impacts of trade: IO models demonstrate a strong positive impact of targeting sectors with lower import rates, contrary to CGE models.

Although the literature estimates that employment impacts will always be stronger in IO than in CGE due to the absence of feedback in the former, we show that the price of capital can lead to weaker effects in IO. Considering the source of the employment effect is thus decisive to undertanding the employment impacts and divergences between the two models.
From a more quantitative viewpoint, investing in solar panels or weatherproofing generates employment in both models. The results are quite similar across models for solar panels. They are higher with IO for weatherproofing, but by a factor from 1.19 to 1.87, a discrepancy much lower than other estimates in the literature. 

The remainder of this paper is organised as follows.
We start with a review of literature in section \ref{sec:literature}. In section \ref{sec:intuition}, we consider a simple IO model, in order to highlight that three mechanisms determine the impact of policy on jobs in IO analysis: the share of labour in value added, wages and rates of imports. We comment on the economic intuitions behind these three potential levers for job creation.
Then, in section  \ref{sec:smallModels}, we discuss the impact of each of these three mechanisms with stylised CGE models.
We go on to give more quantitative estimates in section \ref{sec:fullModel}, using a fully-fledged CGE with 58 sectors and comparing it with IO results.
Finally, section \ref{sec:conclusion} concludes and discusses implications for policy and future research.


%%%%%%

\section{Literature} \label{sec:literature}

Economists have favoured the use of taxes to correct externalities since the work of \citet{Pigou1920}.
But the search for a «~double dividend~», as coined by \citet{Pearce1991}, started with the idea of a carbon tax to mitigate climate change. The initial idea was to tax the negative externalities, i.e. carbon emissions, and use the revenues to reduce more distortionary taxes, thus improving the overall economic performance and reducing pollution at the same time. If the positive impacts ("the revenue-recycling effect") could outweigh the negative impacts ("the tax-interaction effects"), then a "strong" double dividend could be obtained, sensu \citet{Goulder1994}.
The appeal of such a no-regret policy has motivated numerous pieces of theoretical and empirical research. 
In a seminal paper, \citet{Bovenberg1994a} argued that, in a first-best setting, environmental taxes exacerbate distortions in the labour market and in the commodity market, leading to a decrease in employment.
Two years later, \citet{Bovenberg1996} studied the case with unemployment through a fixed wage, and with an unspecified fixed factor. They concluded that a tax on pollution can boost employment if labour is a better substitute for the polluting input than capital. 
An early review of the numerous works on the double dividend was made by \citet{Goulder1994}. \citet{Chiroleu-Assouline2001} provides a summary of theoretical work, and \citet{Patuelli2005} offer a more recent review of 61 quantitative studies on this topic.

This literature has evolved with increasing concerns about global warming and rising unemployment. The employment dividend has become more important. An abundant stream of research has studied whether the energy transition can kill two birds with one stone, through the potential for so-called "green jobs".
Another shift has been the increased perception of inter-sectoral impacts.
In the earlier works, the analytical papers \citep{Bovenberg1994, Bovenberg1996} tried to determine under what conditions a better use of green taxes could reduce unemployment, but they considered an economy with one single sector.
However, the need to invest in new, low-carbon infrastructure has motivated research based on the impacts of investment between different sectors. For example, would it be better to invest in retrofitting buildings or in additional energy sources? in domestically-produced wind turbines, or in cheaper gas-fuelled power plants burning imported gas? 
Models with several sectors became dominant in studying the differences between cost structures and import rates.

In this literature on the energy transition and employment, it is possible to distinguish two broad categories of model that have been used to quantify employment impacts: input-output (IO) models and computable general equilibrium (CGE) models.
Work on input-output models includes \citet{Hillebrand2006, Scott2008, deArce2012, Markaki2013, Hartwig2016, Yushchenko2016, Li2016, Garrett2017}.
Among the most prominent work on CGE models is \citet{Lehr2008, Lehr2012, Bohringer2013, Blazejczak2014, Creutzig2014, Duscha2014, Duscha2016}.

IO models can be considered as a corner-case of CGE, i.e. a CGE model with very specific assumptions. But these assumptions are so specific that IO and CGE can also be considered as different model types. IO's best-known assumptions are Leontief production functions and fixed prices \citep{Miller2009}. \citet{Dwyer2005} list five assumptions underpinning IO models: final demand is exogenous; capital, labour and land are endogenous; there are no price-induced substitution effects; government expenditure remains constant; employment is perfectly elastic. 
Another assumption depends on the category of IO model considered. It is important to distinguish two types of IO model: open (or type I) models and closed (or type II) models \citep{Miller2009}. The differences between these two types lies in the treatment of wages and household consumption. These are endogenized in closed models: additional labour revenues lead to increased consumption, in a circular manner. Closed IO models always yield larger effects than the open versions, as endogenizing household revenues create a self-reinforcing effect (positive or negative).

In this paper, we focus on open (type I) models, as they seem more common in the literature. Thus, the remainder of this paper implicitly refers to open models. In this framework, household consumption is exogenous. It is not influenced by labour revenues. 
With all their assumptions, IO models can also be considered as a significant departure from CGE models: by fixing prices, they deviate from the profit-maximising behaviour of the firm in CGEs. By setting exogenous demand, they also differ from the utility-maximising consumer found in CGE models. A possible pitfall of using exogenous investment and fixed prices is the possibility of increasing investment in order to generate employment, without considering how this increase will be financed or the negative impacts on employment at payback time. For example, \citet{Wei2010} estimates the number of jobs per unit of energy produced, thus creating a bias favouring the costlier sources of energy.

Despite these restrictive assumptions, the use of IO models is motivated by a number of comparative advantages over CGEs. The first is their simplicity: IO models are easy and quick to set up. They are easier to combine with technical bottom-up models, for example in the building sector \citep{Scott2008, Yushchenko2016}. Because they are easier to understand, they also avoid the pitfall of being considered as a "black box" which is sometimes the case with large computational models \citep{Faehn2015}.

Both CGE and IO models are used as "policy models", but their results and mechanisms are never compared nor discussed. By way of analogy, it is similar to having two compasses indicating a different North, but not discussing that "detail". Because both models are used, it raises the question of how different their results are, and why. 
If some results are invariant across the two types of models, it would give more confidence in their robustness. 
If some results diverge significantly, it raises the question of which model should be used. Tracking the differences back to some sets of equations  might help to understand the root cause of their difference, and identify circumstances (e.g. economic conditions) in which one model should be preferred to the other, or at least provide some uncertainty range on the values. 

As an analysis of employment related to the energy transition, our work extends the previous analyses which used only one methodology. In particular, we use the same renewable investment vectors as \citet{Garrett2017}, but add the results from CGE modelling, as well as an explanation of the input-output mechanisms of job creation.

As a comparative exercise, we contribute to the small literature on differences between IO and CGE. 
Other papers that have already mentioned the difference in employment impacts found in IO and CGE models are \citet{Partridge1998, OHara2013}. 
\citet{Dwyer2005} offer a discussion and quantitative comparison of IO and CGE models, at the regional and national level, of the impacts of a special event, an Australian Grand Prix. They find that the employment multiplier is 4.64 higher in the IO model. They point out that there is a crowding out in CGE and not in IO, which to a large extent explains these discrepancies. Other quantitative analyses have studied local or regional impacts \citep{Siegfried2000}, but this is not our focus here. 

Our paper is close to these comparative works, and in particular to the analysis made by \citet{Dwyer2005}, but differs on several points. 
The main one is that we go deeper into the discussion of differences between IO and CGE models. 
We first break down the mechanisms of job creation through a shift in final demand in IO models, and show that they boil down to three components: the relative share of labour in value added, wages, and import rates of the sectors affected by the shift. 
Then, we compare each of these mechanisms with a stylised CGE model. 
In this comparison exercise, we show the corresponding equations of IO and CGE models side by side, and discuss each in turn. Finally, we make a quantitative comparison with a full-fledged CGE, and add a discussion about labour market representation, as well as about the trade-related closure.

The topic under scrutiny is also different. Our focus is the energy transition, not a special event such as a Grand Prix or a stadium investment. We use data for investment in low-carbon sectors. These different data might yield new results, since the high labour content of renewable energy and building insulation is often presented as a lever for job creation. Moreover, it provides comparison between CGE and IO models, which are both used in the literature on the employment effects of an energy transition.

Finally, from a methodological point of view, we do not study an increase in investment, but a shift, a reallocation of the same amount of aggregate investment to different sectors. This approach should alleviate the absence of crowding out effects in IO models, and might provide different results. In addition, this question is of interest for policy makers, as it tries to determine whether job creation is possible with a constrained budget.


\section{Concluding Remarks} \label{sec:conclusion}

In this paper, we have looked at the determinants of job creation through a reallocation of investment towards low-carbon sectors.
We have shown that the net impact of such a shift depends on three parameters in input-output modelling: labour share in value added, wages and import rates for each sector.
IO models suggest a positive impact if the shift benefits sectors with a high share of labour in value added, low wages or low import rates.

We have then tested to what extent these conclusions stand when using a CGE framework. 
This comparison exercise yields the following conclusions for each of the three effects:
\begin{itemize}
	\item Labour share: In CGE, targeting sectors with a high labour share also increases employment, but to a lower extent than in IO models. This is the result of two diverging effects. First, because of flexible prices and substitution possibilities towards capital, the increase in labour demand is lower in a CGE than in IO, and leads to a smaller job creation effect. Second, as sectors with a high labour share are also sectors with a low capital share, targeting these sectors in CGE depresses capital prices. This raises the real wage and thus increases labour supply in CGE compared to IO, leading to a larger job gain. These two effects act in opposite directions, but the first one prevails, and the impact of targeting sectors with high labour share is lower in CGE that in IO.
	
	\item Wages: Targeting sectors with low wages has the same positive impact in CGE and IO models. The labour market representation is much simpler in IO, with an assumption of constant wages. But this simple representation yields the same results as more sophisticated ones with a wage curve representation of labour supply, and labour skill variability or imperfect labour mobility to explain sectoral wage differences. Shifting demand generates exactly the same number of jobs, irrespective of the wage curve elasticity or of substitution between the different types of labour skills.
	
	\item Trade: The employment impacts of encouraging sectors with low import rates diverge strongly between the two models: there is no positive impact in CGE, but a positive effect in IO. 
	In IO models, exports are often exogenous and constant, which means that demand from the rest of the world is perfectly inelastic. And the rate of imports is fixed. Targeting sectors with lower import rates improves the trade balance, resulting in onshoring value added and jobs. 
	In contrast, in CGE, demand from the rest of the world is elastic, and the supply of exports competes with domestic use. With these modelling assumptions, the reallocation of final demand does not create employment through trade in CGE. And this result holds true for the two polar cases of trade closure in CGE: an exogenous trade balance or a fixed exchange rate. When targeting sectors with lower import rates in CGE, the reduction in imports translates either into a fall in exports (for the exogenous foreign investment) or a reduction of private consumption (for the exogenous exchange rate).
	The handling of trade is thus a major divergence between IO and CGE models.
\end{itemize}

Finally, we undertook a quantitative analysis of the employment impacts of investing in weatherproofing or  solar panels, taking into account the benefits in terms of reduced residual consumption of power and gas. We used two models, a fully-fledged CGE and an IO model, and ran them with exactly the same data involving 58 sectors.  
Our numerical application indicates that both of these investments have a positive effect on employment, a result which is robust across models. This positive impact is due to a higher share of labour and lower wages in these sectors, compared to the "electricity and gas" sector. The results are roughly similar in IO and CGE for solar, with a discrepancy ratio between 0.83 and 1.34 for the range of parameters considered; for weatherproofing, the results are higher in IO, with a discrepancy ratio ranging from 1.19 to 1.87.

The literature often assumes that employment impacts are higher in IO than in CGE models, because the former does not account for all negative feed-backs \citep{Partridge1998, Dwyer2005, OHara2013}. In this paper, we show that IO models also have one impact that is potentially less favourable than in CGE: the price of capital is fixed in IO, while it might decrease (or increase) in CGE.
Our results also highlight the fact that the divergence between the two models crucially depends on the origin of the economic mechanisms: the wage level, the labour share or the import rate.
Encouraging sectors with a high labour share or low wages generates employment in both CGE and IO models. IO models provide a good approximation to CGE results as to the impacts of wages on employment. They also provide the same direction for the labour-intensity effect, although they provide higher estimates of their impacts. 
But CGE and IO models diverge significantly concerning the importance of import rates. We see no positive impact of targeting sectors with low import rates in a CGE, but a strong positive impact in IO. Trade impacts are thus an important difference between IO and CGE, although overlooked in the literature.
Finally, from a quantitative point of view, employment impacts are similar across the two models for investment in solar energy, and slightly higher in IO for investment in weatherproofing -- with a discrepancy ratio much lower than other literature estimates. IO might thus be a reasonable approximation to CGE models for estimating employment impacts, at least in these two sectors. This result holds if CGE models are based on capital-labour elasticities consistent with literature estimates, but many CGE overestimate this elasticity, which leads to underestimating employment impacts by 17\% to 40\% in our scenarios. 

Our approach suffers from obvious limitations. We do not investigate all the possible modelling specifications of the labour market, be it the Phillips curve, labour search and matching or others. Nevertheless, we believe that the wage curve specification we have examined provides interesting results because this framework is widely used by the CGE modelling community.
Our representation of trade was kept simple. Monetary effects are also absent from our model. Our work should thus be taken on the assumption that the monetary policy does not impact our results. This might hold true if the monetary policy is kept constant, or if it is not directly under the control of the government, as in the Eurozone.

From a policy perspective, our findings indicate that positive employment impacts can be obtained by shifting investment towards labour-intensive or low-paid sectors. Targeting low-carbon sectors with such characteristics might thus yield a double dividend.
But our results also question the relevance of the argument that favouring local sectors, like renewable energies or insulation works, boosts employment by reducing imports. The prominence of this topic in policy communications seems inversely proportional to the robustness of its economic impacts. 