\section{Proofs}

\subsection{Input-output analysis}
\label{app:input_ouput}


For a 2x2 economy, the $\pmb{A^d}$ matrix is equal to:
$$ \pmb{A^d} = 
\begin{pmatrix} 
	\frac{Z11}{P1}  & \frac{Z12}{P2} \\
	\frac{Z21}{P1} & \frac{Z22}{P2} 
\end{pmatrix}
=
\begin{pmatrix} 
a  & b \\
c & d
\end{pmatrix}
$$
so:
$$\pmb{\pmb{I} - \pmb{A^d}} = 
\begin{pmatrix} 
1 - d  & b \\
c & 1 - a
\end{pmatrix}
$$

We define $\Delta$ as $\Delta = det(\pmb{I} - \pmb{A^d})$. Then
$\pmb{Q} = (\pmb{I} - \pmb{A^d})^{-1}$ is equal to
$$\pmb{Q} = \frac{1}{\Delta}
\begin{pmatrix} 
1 - d  & b \\
c & 1 - a
\end{pmatrix}
$$

To get a better intuition of the mechanism, we introduce the value added per unit of production: $(va)_i = (VA_i/Pi)$.
Then, the value added per unit of final demand is equal to:
$$<va> \cdot Q =
\begin{pmatrix} 
va_1 & 0 \\
0 & va_2
\end{pmatrix}
\cdot Q 
= \frac{1}{\Delta}
\begin{pmatrix} 
va_1 \cdot (1-d) & va_1 \cdot b \\
va_2 \cdot c & va_2 \cdot (1-a)
\end{pmatrix}
$$

$$<va> \cdot Q 
= \frac{1}{\Delta}
\begin{pmatrix} 
(1- a-c)\cdot (1-d) & (1-a-c) \cdot b \\
(1-b-d) \cdot c & (1-b-d) \cdot (1-a)
\end{pmatrix}
$$

which simplifies by noting that the sum of the column are equal to 1 :
$$<va> \cdot Q 
= 
\begin{pmatrix} 
1-\theta_1 & \theta_2 \\
\theta_1  & 1-\theta_2
\end{pmatrix}
$$

with $\theta_1 =  \frac{(1-b-d) \cdot c}{(1-a)(1-d)-bc}$ and $\theta_2 = \frac{(1-a-c) \cdot b}{(1-a)(1-d)-bc}$. 
This formula reveals that the Leontief matrix allocates demand between the various sector to generate a value added. For example, an increase of $\delta$ in the demand adressed to sector S1 would generated a value added in sector S1 equal to $(1-\theta_1) \cdot \delta$, and a value added in sector S2 equal to $\theta_1 \cdot \delta$.

We can now estimate the number of job per unit of final demand. To that end, we define the vector $e$ of job per unit of value added by :
$(e_i) = (E_i/VA_i)$, where $E_i$ is the number of jobs in sector i and $VA_i$ the value added in sector i.
The number of job per unit of domestic demand - which we call domestic employment content $\pmb{ce}^d$ - is equal to:
$$(\pmb{ce^d})^t =
\pmb{e}^t \cdot <va> \cdot Q 
= 
\begin{pmatrix} 
e_1 (1-\theta_1) + e_2 \theta_1 ;&
e_1 \theta_2  + e_2 (1-\theta_2)
\end{pmatrix}
$$

This domestic employment content must now be linked to final demand. 
A shift of $\delta$ in final demand from S1 to S2 leads to a change in domestic demand equals to:
$$\pmb{d^d} =
\pmb{<1-\tau_m>} \cdot 
\begin{pmatrix} 
-\delta  \\
\delta
\end{pmatrix} 
=
\begin{pmatrix} 
- (1 - \tau_1) \\
1 - \tau_2
\end{pmatrix} \cdot \delta
$$

\subsubsection{Impacts of shifting investment}
An increase of one million euros in final demand addressed to S2 leads to an increase in jobs equal to:
$$(1-\tau_2) \cdot \left( (1-\theta_2) e_2+ \theta_2 e_1 \right)$$.

An decrease of one million euros in final demand addressed to S1 leads to an increase in jobs equal to:
$$(1-\tau_1) \cdot \left( (1-\theta_1) e_1+ \theta_1 e_2 \right)$$.

A shift in final demand of one million euros from S1 to S2 leads to an increase in employment if and only if:
$$(1-\tau_2) \cdot \left( (1-\theta_2) e_2+ \theta_2 e_1 \right) > (1-\tau_1) \cdot \left( (1-\theta_1) e_1+ \theta_1 e_2 \right)$$

It then useful to note that the right parenthesis in each side is the domestic employment content, a weighted average of direct employment intensity. The above equation can be re-written as:
$$(1-\tau_2) \cdot ce^d_2> (1-\tau_1) \cdot ce^d_1 $$
so a shift in final demand generates jobs if an only if the product of the employment content and the import rate is higher.
But the computation of the employment content shows that the domestic employment content is high if demand generates value added in sector with a high number of jobs per value added.
